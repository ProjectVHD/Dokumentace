Platforma má za účel vytvořit jednoduchý způsob pro dopravní podniky o sdílení jejich dat o polohách vozidel.
Cestující totiž v místech bez rozšířených struktur OpenData k těmto datům nemají přístup.

Abychom mohli zlepšit život v těchto místech, soustředíme se na dopravní podniky, aby nám své data poskytly a zároveň tím zlepšili kvalitu živote v dané lokalitě.
O implementaci webové aplikace se postaráme my a data příjmáme v jakémkoli formátu. (Protože nám stačí jen nějakej konektor, takže to je pro nás jednoduchý....)

Služba nabízí pro Dopravní podniky další způsob komunikace se svými cestujícími o aktuálním dění. Pro dopravce služba nabízí možnost přidat provozní upozornění přímo do aplikace.

\section{Architektura}

\section{Backend}

\section{Webová aplikace}

Webová apliakce je hlavní částí platformy, která živá data o pohybu vozidel komunikuje k cestujícím.

Využívá přitom komunikace s Backendem za pomocí REST + GRAPH QL API, vše za využití protokolů HTTPS, nebo WebSocket.

\subsection{Zobrazování vozidel}
K živému zobrazování dat je třeba několik kroků.

Data jsou nejprve předána z Backendu pomocí GraphQL do aplikace klienta
Následně jsou data přetransformována za účelem zobrazování živé animace pohybu.
Pomocí frameworku React a jejich key id kktin jsou uchováávn aktuální stav DOM a při jejich následné aktualizaci nejdojde k jejich přepsání (tohle je asi na nějakej nadpis v teoretické části)
Využitím interaktivní mapy Leaflet zobrazujeme pro uživatele předpověď pro aktuální pozici vozidel z přijatých dat.


Webová aplikace ve formě jednotlivých služeb je tvořena z jedné Codebase (Engliš?) sdílená napříč dopravci, úpravy a případné vylepšení u jednoho dopravec jsme schopni ve formě modulu nasadit pro všechny dopravce

