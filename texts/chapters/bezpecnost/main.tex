\chapter{Bezpečnost}
Webové aplikace jsou v~dnešní době rozšířeny do~různých odvětví, včetně školství, zdravotnictí a~samozřejmě hojně i~do~běžného života. Toto rozšíření však s~sebou nese i~rizika kybernetických hrozeb a~útoků. Kyberzločinci neustále hledají zranitelná místa ve~webových aplikacích, a~proto je zabezpečení webových aplikací nezbytnou součástí každého procesu vývoje webových aplikací. \par
Bezpečností webových aplikací se~zabývá projekt OWASP \cite{owasp}(Open Web Application Security Project) se~svým seznamem Top 10 \cite{owasp10} nejzávažnějších bezpečnostních rizik webových aplikací. Tento seznam je pravidelně aktualizován, aby odrážel nejnovější trendy v~oblasti bezpečnostních hrozeb pro webové aplikace.\par
Tento projekt se~snaží o~řešení těchto bodů. Zde jsou některá navržená řešení.\par

\begin{itemize}
    \item \textbf{Kryptografické selhání} \cite{crypto}: Backendová i~frontendová část projektu využívají komunikace pomocí šifrovaného protokolu HTTPS. Certifikáty jsou podepsané certifikačními autoritami Let's~Encrypt\cite{letsencrypt} a~Google Trusted Services.
    \item \textbf{Injection} \cite{nosqlinjection}: GraphQL se stará o validitu datových typů při komunikaci mezi frontendem a backendem. Při práci s uživatelským vstupem probíhá důležitá sanitizace, aby se předešlo nosql injection napadení databáze MongoDB.
    \item \textbf{Magic Link Login} \cite{magiclinklogin}: Uživatelé se nepřihlašují pomocí obvyklé kombinace jméno - heslo. Při přihlášení zadávají pouze svůj email. Na tento email je jim následně odeslán časově omezený klíč pro jednorázové přihlášení. Dochází k eliminaci hrozby uniknutí hesla, jelikož není nikde uloženo.
\end{itemize}