\chapter{Bezpečnost}
Webové aplikace jsou v dnešní době rozšířeny do různých odvětví, včetně školství, zdravotnictí a samozřejmě hojně i do běžného života. Toto rozšíření však s sebou nese i rizika kybernetických hrozeb a útoků. Kyberzločinci neustále hledají zranitelná místa ve webových aplikacích, a proto je zabezpečení webových aplikací nezbytnou součástí každého procesu vývoje webových aplikací. \par
Bezpečností webových aplikací se zabývá projekt OWASP \cite{owasp}(Open Web Application Security Project) se svým seznamem Top 10 nejzávažnějších \cite{owasp10} bezpečnostních rizik webových aplikací. Tento seznam je pravidelně aktualizován, aby odrážel nejnovější trendy v oblasti bezpečnostních hrozeb pro webové aplikace.\par
Tento projekt se snaží o řešení těchto bodů. Zde jsou některá navržená řešení.\par
\section{Kryptografické selhání}\cite{crypto}
Backendová i frontendová část projektu využívají komunikace pomocí šifrovaného protokolu HTTPS. Certifikáty jsou podepsané certifikačními autoritami Let's Encrypt a Google Trusted Services.
\section{Injection}