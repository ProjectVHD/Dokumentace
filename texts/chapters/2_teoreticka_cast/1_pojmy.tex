\section{Pojmy}

\subsection{WebSocket}
WebSocket je technologie pro obousměrnou komunikaci mezi webovým prohlížečem a serverem. Umožňuje otevření komunikačního kanálu mezi klientem a serverem, přes který mohou obě strany posílat a přijímat zprávy v reálném čase. To je velmi užitečné pro aplikace, které potřebují získávat aktualizace nebo zprávy od serveru bez nutnosti, aby klient musel ručně požadovat nové informace. WebSocket poskytuje efektivnější způsob komunikace než technologie jako HTTP, které vyžadují, aby klient a server navazovali nové spojení pro každou zprávu.
\cite{websocket}
\subsection{Frontend}
Vývoj frontendu se týká tvorby uživatelského rozhraní a uživatelského prostředí webových stránek nebo aplikací. Jedná se o část vývoje webových stránek, která se zaměřuje na klientskou stranu neboli část aplikace, která je viditelná pro uživatele a komunikuje s ním. Frontendoví vývojáři používají jazyky jako HTML, CSS a JavaScript k vytvoření vizuálního rozvržení, designu a funkčnosti webových stránek nebo aplikace. Používají také frameworky, jako jsou React, Angular a Vue.js, aby uspořádali a strukturovali svůj kód, který je tak efektivnější a škálovatelnější.
\subsection{Backend}
Vývojem backendu se rozumí vytváření serverové části webových stránek nebo aplikací. Jedná se o část vývoje webu, která se zaměřuje na logiku a funkce, jež pohání aplikaci, a je zodpovědná za zpracování a manipulaci s daty a komunikaci s frontendem. Vývojáři backendu používají jazyky jako Java, NodeJS a Go, aby vytvořili logiku a funkčnost aplikace. \par
Využívají se frameworky jako Express pro NodeJS, nebo Chi pro Go, které usnadňují vývoj backendu a zvyšují jeho efektivitu a škálovatelnost.
\subsection{Framework}
Framework je sada předpřipraveného kódu a nástrojů, které poskytují strukturu pro vývoj určitého typu aplikace nebo softwaru. Je navržen tak, aby vývojářům ušetřil čas a námahu tím, že jim poskytne základní strukturu a společné funkce a umožní jim soustředit se na vytváření jedinečných funkcí aplikace. Frameworky lze použít pro vývoj frontendů i backendů a k dispozici je mnoho různých typů frameworků. \par
Při použití frameworku mohou vývojáři využívat předpřipravené moduly a knihovny, což pomáhá zkrátit dobu vývoje, zlepšit udržovatelnost kódu a zvýšit celkový výkon a škálovatelnost aplikace. Frameworky také často přicházejí se sadou konvencí, osvědčených postupů a dokumentace, které pomáhají zajistit, aby byl kód uspořádaný a konzistentní, což může usnadnit jeho pochopení a údržbu ostatními vývojáři.
\subsection{Databáze}
Databáze je systém pro ukládání a správu dat.
\subsection{Cache}\label{cache}
Cache neboli mezipaměť je termín, který označuje dočasné uložení dat, aby bylo můžné k nim později přistupovat rychleji. Cache může zlepšit výkon aplikace, protože umožňuje ukládat data, která se často používají, do paměti a pak je při dalším použití získat rychleji. \par
\subsection[JDF]{Jednotný datový formát JDF}\label{JDF}
Jednotný datový formát JDF je sada pravidel vydaných v metodickém pokynu č.5 k organizaci celostátního informačního systému o jízdních řádech vydaném Ministerstvem dopravy ČR. JDF udává strukturu souborů a jejich obsah. Je založen na formátu CSV (Comma Separated Values) - záznamově orientovaný formát dat s oddělovači (pole oddělena čárkou, záznamy odděleny středníkem a CRLF). Aktuálně existují tři verze JDF: JDF 1.9, JDF 1.10 a JDF 1.11, každá využívaná různými dopravními podniky. \par