\section{Technologie}

\subsection{JavaScript}

JavaScript je programovací jazyk, který je používán především pro vývoj webových aplikací. Je to jazyk s dynamickým typováním, což znamená, že se datové typy proměňují v závislosti na hodnotách, které jsou jim přiřazeny. JavaScript může být použit pro tvorbu interaktivních webových stránek, včetně těch, které reagují na události, jako jsou kliknutí na tlačítka nebo přetažení objektů. Je také používán pro tvorbu mobilních aplikací a pro spouštění úloh na straně serveru pomocí technologie Node.js. JavaScript je jedním z nejčastěji používaných programovacích jazyků na webu.

%% vysv2tlit just in time compilation
\subsection{TypeScript}
TypeScript je programovací jazyk, který je postaven na JavaScriptu a přidává do něj statické typování. To znamená, že v TypeScriptu je nutné explicitně určit datové typy proměnných a funkcí, což může pomoci při vyhledávání chyb v kódu a umožňuje lepší předvídatelnost a spolehlivost programů. TypeScript je navržen tak, aby podporoval rozšiřitelnost a zjednodušil vývoj velkých aplikací. Jeho kód se kompiluje do JavaScriptu, takže může být použit na všech místech, kde je podporován JavaScript. TypeScript je používán pro vývoj webových aplikací a je často používán spolu s frameworky jako Angular nebo React.js.
\subsection{React.js}
React.js je populární knihovna jazyka JavaScript pro vytváření uživatelských rozhraní. Umožňuje vývojářům vytvářet opakovaně použitelné, složitelné komponenty, které lze snadno vykreslovat a aktualizovat v reakci na změny dat nebo interakce s uživatelem. Jednou z klíčových vlastností Reactu je jeho schopnost efektivně aktualizovat uživatelské rozhraní v reakci na změny v podkladových datech, což se děje prostřednictvím procesu zvaného "virtuální rozdílení DOM". To může vést k výraznému zvýšení výkonu, zejména u rozsáhlých a složitých aplikací. React také poskytuje užitečnou sadu vývojářských nástrojů a vývojářsky přívětivých funkcí, jako je JSX, syntaktické rozšíření jazyka JavaScript, které umožňuje vkládat do kódu jazyka JavaScript prvky podobné HTML. Popularita Reactu v posledních letech rychle roste a je široce používán v mnoha populárních webových stránkách a webových aplikacích, jako jsou Facebook, Instagram, Netflix a Airbnb. React má také velkou a aktivní komunitu vývojářů, kteří přispívají k vývoji knihovny a vytvářejí mnoho knihoven a nástrojů třetích stran, které lze použít k vylepšení možností vývoje Reactu.
\subsection{Go}
Go je programovací jazyk, který byl vyvinut společností Google. Je to staticky typovaný jazyk, který je navržený tak, aby byl efektivní a snadno použitelný. Go se často používá pro vývoj webových aplikací a nástrojů pro správu a automatizaci systémů. Jeho jednoduchost a rychlost kompilace jsou mezi vývojáři velmi oblíbené.

\subsection{GraphQL}
GraphQL je dotazovací jazyk pro rozhraní API, který klientům umožňuje předvídatelným a efektivním způsobem požadovat přesně ta data, která potřebují. Na rozdíl od jazyka REST, který k získání všech dat potřebných pro konkrétní zobrazení nebo funkci vyžaduje několik koncových bodů, jazyk GraphQL umožňuje vývojářům získat všechna potřebná data jediným požadavkem. To může vést k výraznému zvýšení výkonu, zejména na mobilních zařízeních nebo při práci s velkým množstvím dat. Jazyk GraphQL navíc umožňuje větší flexibilitu ve struktuře vracených dat, protože klient může požadovat přesně ta pole, která potřebuje, a nikoli pevnou sadu polí definovanou rozhraním API. To umožňuje efektivnější využití šířky pásma sítě a může zjednodušit kód klienta.
\subsection{Redis}

\subsection{Docker}