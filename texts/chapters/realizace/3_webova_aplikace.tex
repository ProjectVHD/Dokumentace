\section{Webová aplikace}

Webová apliakce je hlavní částí platformy, která živá data o pohybu vozidel komunikuje k cestujícím.

Využívá přitom komunikace s Backendem za pomocí REST + GRAPH QL API, vše za využití protokolů HTTPS, nebo WebSocket.



Webová aplikace ve formě jednotlivých služeb je tvořena z jedné Codebase (Engliš?) sdílená napříč dopravci, úpravy a případné vylepšení u jednoho dopravec jsme schopni ve formě modulu nasadit pro všechny dopravce


\subsection{Zobrazování vozidel}
K živému zobrazování dat je třeba několik kroků.

Data jsou nejprve předána z Backendu pomocí GraphQL do aplikace klienta
Následně jsou data přetransformována za účelem zobrazování živé animace pohybu.
Pomocí frameworku React a jejich key id kktin jsou uchováávn aktuální stav DOM a při jejich následné aktualizaci nejdojde k jejich přepsání (tohle je asi na nějakej nadpis v teoretické části)
Využitím interaktivní mapy Leaflet zobrazujeme pro uživatele předpověď pro aktuální pozici vozidel z přijatých dat.

\subsubsection{optimalizace výkonu}

\subsection{Detail spojení}
Pro cestující zpracováváme data o jízdních řádech a nabízíme ve webové aplikaci možnost zobrazit si bližší detail o spojení, které vozidlo právě obsluhuje.

Po stisknutí tlačítka se cestujícímu zobrazí blližší informace o spojení, jako je čas příjezdu k jednotlivým zastávkám spoje, nebo předpokládané zpoždění hlášené dopravcem.

Tyto informace jsou k dispozici ve formátu JDF a do aplikace klienta předány pomocí GRAPHQL dotazu na Backend.

\subsection{Provozní upozornění}
Dopravci mohou přidávat do aplikace klienta provozní upozornění, které se zobrazí v případě, že se vyskytne nějaká situace, která může mít vliv na cestu cestujícího.
\subsection{Práce s polohou uživatele}
\subsection{Monitorování návštěvnosti}
\subsection{PWA}