\section{Webová aplikace}

Frontendová část platformy, ve formě webové aplikace a aplikace mobilní je hlavní částí platformy, která živá data o pohybu vozidel komunikuje k cestujícím.

Využívá přitom komunikace s Backendem za pomocí REST + GraphQL API, vše za využití protokolů HTTPS, nebo WebSocket.

Webová aplikace využívá technologie NextJS pro statické generování aplikace na straně serveru (pro podporu SEO).
Webová aplikace využívá za svého chodu spoustu doplňujících dotazů přes internet pro dosažení interaktivity, takzvaných XHR requestů.

Webová stránka díky tomu odpovídá jednostránkové aplikaci s ovládacími prvky.

\subsection{Mapa}
Byla vytvořena webová aplikace pro zobrazování aktuálních poloh vozidel dopravních podniků a jiných dopravců.
Na jednotné mapě zobrazujeme všechny dopravce, se kterými spolupracujeme.
Mapu lze ovládat na mobilních zařízeních díky pohybům prsem, nebo na stolních počítačích myší. Podle uživatelského pohledu aplikace vybere dopravce, které by měla zobrazovat a dochází zde k optimalizaci přenášených dat. Do zařízení uživatele jsou přenášeny pouze polohy vozidel, které na mapě vidí.

Pro přehlednost mapa implementuje funkci shlukování vozidel do bublin. Shlukování do bublin zajistí v aplikaci přehlednost a plní i funkci optimalizace výkonu mapy.
Při vzdáleném pohledu je funkce plynulé animace pohybu vozidel vypnuta.

\subsection{Zobrazování vozidel}
K živému zobrazování dat se využívá GraphQL\ref{graphql} subscription "GetVehicles", díky které z backendu automaticky přijde zpráva s aktuálními daty. Tato data však mívají rozestup 10 sekund, proto se animace plynulého přechodu dopočítává. Jednotlivá data vozidel jsou rozdělena podle dopravců, subscription\ref{websocket} se tedy upravuje podle toho, kam se uživatel zrovna dívá.
\begin{figure}[H]
    \centering
    \includegraphics[width=1\textwidth]{images/Screenshot from 2023-03-18 20-15-00.png}
    \caption{Ukázka části mapy}
    \label{mapa}
\end{figure}

Pro dosažení plynulosti zobrazování využívá aplikace nejlepších praktik pro práci s frameworkem ReactJS\ref{reactjs}. Časování animace přenechává aplikace webovému prohlížeči uživatele, využívá totiž Webového rozhraní Window.requestAnimationFrame()\cite{animationframe}, které zajišťuje vytváření nových snímků ve vhodné frekvenci.
\subsection{Cache}
Data prostřednictvím dotazování Backendu jsou na straně cestujícího ukládána do cache, aby již nebylo třeba tato data opakovaně načítat.
Cache\ref{cache} na straně webové aplikace šetří uživatelům jejich datové připojení a zároveň snižují nároky na vytěžování serveru.

\subsection{Detail spojení}
Uživatel má po kliknutí na jedoucí spoj k dispozici základní informace o tomto spoji, číslo linky zpoždění, další zastávka a cílová zastávka. Toto vše je vždy individálně upravitelné pro každý dopravní podnik. Po rozkliknutí bližšího detailu o spojení, se uživateli zobrazí dodatečné informace viz obr.\ref{detail7}

\begin{figure}[H]
    \centering
    \includegraphics[width=0.6\textwidth]{images/global_pce_con_detail_7.png}
    \caption{Detail linky 7}
    \label{detail7}
\end{figure}

Současně s zobrazením bližších informací o spoji se zobrazuje také grafická reprezentace trasy, kudy daný spoj pojede, včetně zobrazení zastávek.

\begin{figure}[H]
    \centering
    \includegraphics[width=0.7\textwidth]{images/krom_line_6.png}
    \caption{Detail trasy v Kroměříži}
    \label{trasa}
\end{figure}

\subsection{Provozní upozornění}
Dopravci mohou přidávat do aplikace klienta provozní upozornění, které se zobrazí v případě, že se vyskytne nějaká situace, která může mít vliv na cestu cestujícího.\par
Za tímto účelem vzniklo uživatelské prostředí pro dispečink dopravce.

Aplikace pravidelně zjistí aktuální stav a zobrazí nejnovější informace.

\par
\begin{figure}[H]
    \centering
    \includegraphics[width=0.75\textwidth]{images/global_arriva_event.png}
    \caption{Ukázka provozního upozornění}
    \label{upozorneni}
\end{figure}
\subsection{Práce s polohou uživatele}
Aplikace na požádání od uživatele získá jeho aktuální polohu a pro cestujícího v zobrazí zastávky v jeho blízkosti.

\begin{figure}[H]
    \centering
    \includegraphics[width=0.75\textwidth]{images/position.png}
    \caption{Ukázka polohy uživatele}
    \label{poloha}
\end{figure}
\subsection{Monitorování návštěvnosti}
Zajímavou funkcí pro dopravce je monitorování návštěvnosti aplikace za pomocí Google Analytics. Pro získávání přesných dat o užitečných a oblíbených funkcích aplikace využíváme monitorování vlastních událostí, např. při otevření detailu spoje, nebo vyhledání detailu zastávky.
\subsection{Aplikace a web}
Aplikace využívá experimentálních funkcí technologie PWA\ref{pwa} jako je například načítání GPS souřadnice uživatele, zobrazování notifikací, nastavení připomínky na přijíždějící spoj.
Funkce PWA zároveň nabízí stahování map do zařízení, aby nemuseli být při každém spuštění znovu stahovány.

Díky technologii PWA je webové aplikaci umožněno využít vnitřního uložiště zařízení pro ukládání dat o mapách, a tím je možné ušetřit uživatelům další stahování a prodlevu před možností využívání aplikace.
