\section{Webová aplikace}

Webová apliakce je hlavní částí platformy, která živá data o pohybu vozidel komunikuje k cestujícím.

Využívá přitom komunikace s Backendem za pomocí REST + GRAPH QL API, vše za využití protokolů HTTPS, nebo WebSocket.



Webová aplikace ve formě jednotlivých služeb je tvořena z jedné Codebase (Engliš?) sdílená napříč dopravci, úpravy a případné vylepšení u jednoho dopravec jsme schopni ve formě modulu nasadit pro všechny dopravce

\subsection{Mapa}
Byla vytvořena webová aplikace pro zobrazování aktuálních poloh vozidel dopravních podniků a jiných dopravců.
Na jednotné mapě zobrazujeme všechny dopravce, se kterými spolupracujeme.
Mapu lze ovládat na mobilních zařízeních díky pohybům prsem, nebo na stolních počítačích myší. Podle uživatelského pohledu aplikace vybere dopravce, které by měla zobrazovat a dochází zde k optimalizaci přenášených dat. Do zařízení uživatele jsou přenášeny pouze polohy vozidel, které na mapě vidí.

\subsection{Zobrazování vozidel}
K živému zobrazování dat je třeba několik kroků.

Data jsou nejprve předána z Backendu pomocí GraphQL do aplikace klienta
Následně jsou data přetransformována za účelem zobrazování živé animace pohybu.
Pomocí frameworku React a jejich key id kktin jsou uchováávn aktuální stav DOM a při jejich následné aktualizaci nejdojde k jejich přepsání (tohle je asi na nějakej nadpis v teoretické části)
Využitím interaktivní mapy Leaflet zobrazujeme pro uživatele předpověď pro aktuální pozici vozidel z přijatých dat.

\subsection{Detail spojení}
Pro cestující zpracováváme data o jízdních řádech a nabízíme ve webové aplikaci možnost zobrazit si bližší detail o spojení, které vozidlo právě obsluhuje.

Po stisknutí tlačítka se cestujícímu zobrazí blližší informace o spojení, jako je čas příjezdu k jednotlivým zastávkám spoje, nebo předpokládané zpoždění hlášené dopravcem.

Tyto informace jsou k dispozici ve formátu JDF a do aplikace klienta předány pomocí GRAPHQL dotazu na Backend.

\subsection{Provozní upozornění}
Dopravci mohou přidávat do aplikace klienta provozní upozornění, které se zobrazí v případě, že se vyskytne nějaká situace, která může mít vliv na cestu cestujícího.
\subsection{Práce s polohou uživatele}
\subsection{Monitorování návštěvnosti}
Zajímavou funkcí pro dopravce je monitorování návštěvnosti aplikace za pomocí Google Analytics. Pro získávání přesných dat o užitečných a oblíbených funkcích aplikace využíváme monitorování vlastních událostí, např. při otevření detailu spoje, nebo vyhledání detailu zastávky.
\subsection{PWA}

 /// část do technologíí
Technologie PWA (Progresivní webová aplikace) je technologická novinka. Do budoucna by se měla stát jednou z nejlepších možností vývoje aplikací. (Tady se mega hodí nějakej random článek jako zdroj)
Tuto technologii jsme se rozhodli využít díky její možnosti adaptace do všech zařízení s moderními webovými prohlížeči.

 /// čast do realizace
Aplikace využívá experimentálních funkcí nové technologie PWA (Progresivní webové aplikace) jako je například načítání GPS souřadnice uživatele, zobrazování notifikací, nastavení připomínky na přijíždějící spoj.
Funkce PWA zároveň nabízí stahování map do zařízení, aby nemuseli být při každém spuštění znovu stahovány.