\section{Backend}
Backendová část projektu má za~úkol získat data od~jednotlivých dopravních podniků, získat a zpracovat data o~jízdních řádech \cite{cisjr} a~za~pomoci GraphQL je zprostředkovat frontendové části.
\subsection{Využité technologie} Backendová část projektu je napsaná v~programovacím jazyku GO. Routovaní zajišťuje jednoduchá knihovna Chi-go. Hlavním endpointem celé aplikace je “\query{/query}“, za~který odpovídá GraphQL, jenž se~stará o~dotazy a~subscriptions. Díky využití GraphQL subscriptions je frontend schopen zobrazovat data v~téměř reálném čase. Jako hlavní databázový systém byl využit MongoDB \ref{mongo}, kvůli flexibilitě a~rychlé škálovatelnosti. Pro zrychlení opakujících se~operací, aplikace využívá Redis \ref{redis} jako rychle dostupnou cache. \par
\subsection{Služby}
Samotný backend využívá návrhovou architekturu mikroslužeb (microservice). Je rozdělen na~několik menších částí, které mají za~úkol jiné operace. První microservice se~zabývá stahováním a~zpracováváním dat pro strojové čtení jízdních řádů. V~aktuální době však neexistuje žádné podobné nařízení, které by upravovalo zveřejňování dat o~aktuálních polohách vozidel. Nejednotnost těchto dat znamená, že pro každý dopravní podnik musela být vytvořena microservice s~individuálním konektorem. Poslední důležitou službou je samotná komunikace s~frontendovou částí.
\subsection [Jízdní řády]{Strojové čtení jízdních řádů} \label{strojoveCteniJR}
Podle vyhlášky \cite{vyhlaskaJizdniRady} musí jednotlivé dopravní podniky zveřejňovat jízdní řády spojů. Jedna microservice se~zabývá stahováním a~zpracováváním těchto dat. \par
Každý dopravní podnik má povinnost zveřejňovat své jízdní řády pro strojové čtení. Tyto data dopravní podniky publikují prostřednictvím Celostátního informačního systému o~jízdních řádech(CISJŘ) \cite{cisjr}. Tento systém spravuje firma CHAPS spol. s.r.o. \cite{chaps}. Výstupem CISJŘ je veřejně přístupný ftp server na~adrese 'ftp://ftp.cisjr.cz'. Data jsou zveřejňována ve~formátu JDF \ref{JDF} a~jsou zazipována. \par
Ze struktury a~dat těchto souborů microservice pro zpracování jízdních řádů vytvoří v~databázovém systému MongoDB dokumenty jednotlivých spojení veřejné dopravy. Tento dokument obsahuje informace o~příjezdech a~odjezdech ze~zastávek, jejich pevné kódy, typ vozidla a~další. Jednotlivé spojení veřejné dopravy jsou jednoznačně identifikovatelná za~pomoci kombinace čísla linky a~čísla spoje.
\subsection {Data dopravních podniků}
Dopravní podniky z~pravidla využívají data o~aktuální poloze vozidel pouze pro interní použití na~dopravních dispečincích. Tato data nejsou nijak upravována nařízeními nebo vyhláškami, což znamená že nejsou sjednocená pravidla mezi podniky, jak by data měla vypadat.\par
Ze strany této aplikace to znamená nutnost implementace individuální microservice pro každý dopravní podnik. Úkolem těchto microservis je získání dat o~aktuální poloze vozidel a~jejich zpracování do~jednotného formátu pro použití. Data jsou poté ukládána do~datového úložiště Redis \ref{redis}, které funguje jako cache \ref{cache}, odkud je hlavní microservice čte.\par
Od dopravních podniků Informace o~poloze spoje se~také ukládá v~kontextu celé trasy, což umožňuje cestujícím vidět přesnou trasu spoje, včetně informací o~zastávkách.
\subsection {Komunikace s~frontendem}\label{mainBackend}
Poslední microservice má za~úkol poskytovat všechna získaná data frontendové části. K~tomuto úkolu bylo nasazeno rozhraní GraphQL. Správné využití GraphQL zabraňuje přebytečnému stahování velkého množství dat a~pomáhá ke~zjednodušení komunikace. \cite{graphqleasy} \par
Hlavními informacemi, které backend poskytuje frontendové části, jsou aktuální polohy vozidel. K~tomuto účelu slouží subscription „\query{GetVehicles}“, která každých 10~sekund pošle nová data.\par
Pro získání dodatečných informací o~spoji slouží query “\query{GetConnectionDetail}“ \newline a~“\query{GetRoads}“, které potřebují dva argumenty, číslo linky “\query{line\_name}“ \newline a~číslo spoje “\query{connection\_number}“ k~jedinečné identifikaci spoje.\par
K~získání budoucích odjezdů ze~zastávek slouží query „\query{GetStationConnections}“, která potřebuje jako argument “\query{IDzastávky}“ a~vrací zpět spoje které v~příštích 45~minutách projedou danou zastávkou.
\newpage
\subsection{Nasazení na~server}
Platforma je nasazena za~pomoci kontejnerizace. Kontejnerizace nám zaručuje identické prostředí, které obsahuje pouze nezbytná data k~běhu jednotlivých microservice.\par
Pro vytvoření kontejnerů se~využívá Docker. Proces vytváření jednotlivých kontejnerů zahrnuje použití souboru „\query{Dockerfile}“, který přesně popisuje jednotlivé kroky k~sestavení.\par
K~následnému nasazení je využíván docker-compose, který umožňuje definovat služby (jednotlivé microservice, mongoDB, redis) a~komunikaci mezi službami. Pro definování služeb se~využívá soubor „\query{docker-compose.yaml}“, který poté dovoluje spravovat všechny kontejnery pomocí jediného příkazu.
