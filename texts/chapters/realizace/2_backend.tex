\section{Backend}
Backendová část projektu má za~úkol získat data od~jednotlivých dopravních podniků, získat a zpracovat data o~jízdních řádech \cite{cisjr} a~za~pomoci GraphQL je zprostředkovat frontendové části.
\subsection{Využité technologie} Backendová část projektu je napsaná v~programovacím jazyku GO. Routovaní zajišťuje jednoduchá knihovna Chi-go. Hlavním endpointem celé aplikace je “\query{/query}“, za~který odpovídá GraphQL, jenž se~stará o~dotazy a~subscriptions. Díky využití GraphQL subscriptions je frontend schopen zobrazovat data v~téměř reálném čase. Jako hlavní databázový systém byla využita databáze MongoDB \ref{mongo}, díky své flexibilitě a~rychlé škálovatelnosti. Pro urychlení často-se opakujících operací, aplikace využívá Redis \ref{redis}, rychle dostupnou cache. \par
\subsection{Služby}
Samotný backend využívá návrhovou architekturu mikroslužeb (microservice). Je rozdělen na~několik menších částí, které mají za~úkol různé operace. Podle jejich funkce je můžeme rozdelit do následujících kategorií:

\begin{itemize}
    \item Microservice zabývající se~aktualizací a~zpracováváním dat pro strojové čtení jízdních řádů. K jejímu spuštění dochází pravidelně každý den, pro zajištění aktuálních dat.
    \item Mikroservice zajišťující komunikaci s dopravci, jejich cílem je načítání dat o polohách vozidel v reálném čase. Data však bývají nejednotná a je třeba individuální implementace pro každý dopravní podnik.
    \item Služba obstarávající samotnou komunikaci s~frontendovou částí. Pro aplikaci zajišťuje dostupnost správných dat.
\end{itemize}
\subsection [Jízdní řády]{Strojové čtení jízdních řádů} \label{strojoveCteniJR}
Podle vyhlášky \cite{vyhlaskaJizdniRady} má každý dopravní podnik povinnost zveřejňovat jízdní řády svých spojů ve formátu, vhodném pro strojové čtení. Tyto data dopravní podniky publikují prostřednictvím Celostátního informačního systému o~jízdních řádech(CISJŘ) \cite{cisjr}. Tento systém spravuje firma CHAPS spol. s.r.o. \cite{chaps}. Výstupem CISJŘ je veřejně přístupný ftp server na~adrese 'ftp://ftp.cisjr.cz'. Data jsou zveřejňována ve~formátu JDF \ref{JDF} a~jsou zazipována. \par
Pro účel aktualizace a zpracování těchto dat slouží samostatná microservice “\query{JDF Parser}“.
Ze struktury a~dat těchto souborů microservice vytvoří v~databázovém systému MongoDB dokumenty jednotlivých spojení veřejné dopravy. Tento dokument obsahuje informace o~příjezdech a~odjezdech ze~zastávek, jejich pevné kódy, typ vozidla a~další. Jednotlivé spojení veřejné dopravy jsou jednoznačně identifikovatelná za~pomoci kombinace čísla linky a~čísla spoje.
\subsection {Data dopravních podniků}
Dopravní podniky zpravidla využívají data o~aktuální poloze vozidel pouze pro interní použití na~dopravních dispečincích. Tato data nejsou nijak upravována nařízeními nebo vyhláškami, což znamená že nejsou sjednocená pravidla mezi podniky, jak by data měla vypadat.\par
Z tohoto faktu vznikla nutnost implementace individuální microservice pro každý dopravní podnik. Úkolem těchto microservis je získání dat o~aktuální poloze vozidel a~jejich zpracování do~jednotného formátu pro použití. Data jsou poté ukládána do~datového úložiště Redis \ref{redis}, které funguje jako cache \ref{cache}, odkud je hlavní microservice čte.\par
Informace o~poloze spoje jsou ukládána v~kontextu celé trasy, což umožňuje cestujícím vidět přesnou trasu spoje, včetně informací o~zastávkách.
\subsection {Komunikace s~frontendem}\label{mainBackend}
Poslední microservice má za~úkol poskytovat všechna získaná data frontendové části. K~tomuto úkolu bylo využito rozhraní GraphQL. Správné využití GraphQL zabraňuje přebytečnému stahování velkého množství dat a~pomáhá ke~zjednodušení komunikace. \cite{graphqleasy} \par
Hlavními informacemi, které backend poskytuje frontendové části, jsou aktuální polohy vozidel. K~tomuto účelu slouží subscription „\query{GetVehicles}“, která každých 10~sekund pošle nová data.\par
Pro získání dodatečných informací o~spoji slouží query “\query{GetConnectionDetail}“ \newline a~“\query{GetRoads}“, které potřebují dva argumenty, číslo linky “\query{line\_name}“ \newline a~číslo spoje “\query{connection\_number}“ k~jedinečné identifikaci spoje.\par
K~získání budoucích odjezdů ze~zastávek slouží query „\query{GetStationConnections}“, která potřebuje jako argument “\query{IDzastávky}“ a~vrací zpět spoje které v~příštích 45~minutách projedou danou zastávkou.
\subsection{Nasazení na~server}
Platforma je nasazena za~pomoci kontejnerizace. Kontejnerizace nám zaručuje identické prostředí, které obsahuje pouze nezbytná data k~běhu jednotlivých microservice.\par
Pro vytvoření kontejnerů se~využívá Docker. Proces vytváření jednotlivých kontejnerů zahrnuje použití souboru „\query{Dockerfile}“, který přesně popisuje jednotlivé kroky k~sestavení.\par
K~následnému nasazení je využíván docker-compose, který umožňuje definovat služby (jednotlivé microservice, mongoDB, redis) a~komunikaci mezi službami. Pro definování služeb se~využívá soubor „\query{docker-compose.yaml}“, který poté dovoluje spravovat všechny kontejnery pomocí jediného příkazu.
\newpage