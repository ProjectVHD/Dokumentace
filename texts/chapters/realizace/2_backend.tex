
\section{Backend}
Backendová část projektu má za úkol získat data od jednotlivých dopravních podniků, získat data z Celostátního informačního systému o jízdních řádech\cite{cisjr} a za pomoci GraphQL je zprostředkovat frontendové části.
\subsection{Využité technologie}
Backendová část projektu je napsaná v programovacího jazyka GO. Routovaní zajišťuje jednoduchá knihovna Chi-go. Hlavním endpointem celé aplikace je "/query", za který odpovídá GraphQL, jenž se stará o dotazy a subscriptions. Díky využití GraphQL subscriptions je frontend schopen zobrazovat data v téměř reálném čase.\par

/*golang, redis, mongo, docker, graphql, clouflare tunnels*/
\subsection{několik služeb}
Samotný backend je rozdělen na několik menších částí, které mají za úkol jiné operace. Tato návrhová architektura se nazývá mikroslužby (microservices). Podle vyhlášky\cite{vyhlaskaJizdniRady} musí jednotlivé dopravní podniky zveřejňovat jízdní řády spojů. Jedna microservice se zabývá stahováním a zpracováváním těchto dat. \par
V aktuální době však neexistuje žádné podobné nařízení, které by upravovalo zveřejňování dat o aktuálních polohách vozidel. Nejednotnost těchto dat znamená, že pro každý dopravní podnik musela být vytvořena microservice s individuálním konektorem. Poslední důležitou službou je samotná komunikace s frontendovou částí.
\subsubsection{jednoduchost přidání dalšího dopravce}
modularita
\subsection {komunikace s datovými zdroji}
/*data jizdnich řadu z cisjr*/
\subsection {komunikace s jednotlivými podniky}
/*V aktuální době však neexistuje žádné podobné nařízení, které by upravovalo zveřejňování dat o aktuálních polohách vozidel. Jednostlivé dopravní podniky však mají většinou dispečink, ze kterého mohou aktuální polohu vozidel monitorovat. Existence dopravních dispečinků napovídá že tato
každý dopravce má svůj vlastní přístup k datům*/
\subsection {získávání dat spojů}
přiřazení aktuálně jedoucích vozidel k odpovídajícím statickým spojům
\subsection {cache}
cache dat o poloze a nejčastějších spojích a zastávkách
\subsection {graphql}
Subscriptions a query
\subsection{Nasazení na server}
