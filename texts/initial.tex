\begin{titlepage}
  \bfseries{
    \begin{center}
      \LARGE{STŘEDOŠKOLSKÁ ODBORNÁ ČINNOST}

      \vspace{14pt}
      \large{
        \specialization
      }

      \vspace{0.4 \textheight}

      \LARGE{
        \mainTitle
      }
      \vspace{0.3 \textheight}
    \end{center}

    \noindent\Large{\authorName}

    \noindent\Large{\district \hspace{\stretch{1}} \city, \publicationYear}
  }
\end{titlepage}

{\bfseries
\begin{center}
  \LARGE{STŘEDOŠKOLSKÁ ODBORNÁ ČINNOST}

  \vspace{14pt}
  {\large
    \specialization
  }

  \vspace{0.3 \textheight}

  \LARGE{
    \mainTitle
  }

  \LARGE{
    \mainTitleEN
  }

  \vspace{0.2\textheight}
\end{center}
}
{\Large
\noindent\textbf{Autoři:} \authorName\\
\textbf{Škola:} \school\\
\textbf{Kraj:} \district\\
\textbf{Konzultant: } \consultant\\
}

\noindent \city, \publicationYear

\cleardoublepage

\noindent{\Large{\bfseries{Prohlášení}}}

\noindent Prohlašujeme, že jsme svou práci SOČ vypracovali samostatně a použili jsme pouze prameny a literaturu uvedené v seznamu bibliografických záznamů.

\noindent Prohlašujeme, že tištěná verze a elektronická verze soutěžní práce SOČ jsou shodné.

\noindent Nemáme závažný důvod proti zpřístupňování této práce v souladu se zákonem č. 121/2000 Sb., o právu autorském, o právech souvisejících s právem autorským a o změně některých zákonů (autorský zákon) ve znění pozdějších předpisů.

\vspace{24 pt}

\noindent V Pardubicích dne 24. března 2023 \dotfill{}\hspace{\stretch{0.5}}

\hspace{6.45cm} \authorName

\cleardoublepage

\vspace*{0.8\textheight}
\noindent{\Large{\bfseries{Poděkování}}}

\noindent
Chtěli bychom poděkovat Bc. Vlaďce Janů za odborný dohled a vedení technické stránky projektu a panu Ing. Jiří Formánekovi za značnou pomoc v komunikaci s dopravci. %% RNDr. Jan Koupil, Ph.D.
Dále bychom chtěli poděkovat všem, kteří se podíleli na testování platformy.

\cleardoublepage

\noindent{\Large{\bfseries{Abstrakt}}}

\noindent Práce SOČ dokumentuje vývoj platformy pro vizuální zobrazování pohybů vozidel hromadné dopravy. Platforma nabízí aktuální pohled na polohu hromadné dopravy pomocí webové a mobilní aplikace. Realizace projektu zahrnuje backendovou část a webový \\ frontend.

\vspace{18pt}

\noindent{\Large{\bfseries{Klíčová slova}}}

\noindent Programování; GraphQL; Progresivní webová aplikace; Yarn Workspaces; NextJS; Go programovací jazyk
\vspace{18pt}

\noindent{\Large{\bfseries{Abstract}}}

\noindent This thesis documents development of a platform for visual display of public transport vehicles. The platform offers an up-to-date view of the location of public transport using web and mobile application. The project implementation includes a backend and a web frontend.

\vspace{18pt}

\noindent{\Large{\bfseries{Keywords}}}

\noindent Programming; GraphQL; Progressive Web Application; Yarn Workspaces; NextJS; Go programming language

\cleardoublepage