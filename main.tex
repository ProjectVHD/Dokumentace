\documentclass[12pt, a4paper, twoside, openright]{report} %% -- oboustranné

%% Promenne
\newcommand\city{Pardubice}
\newcommand\district{Pardubický kraj}
\newcommand\specialization{Obor č. 18: Informatika}
\newcommand\school{DELTA - Střední škola informatiky a ekonomie, s.r.o.}
\newcommand\publicationYear{2023}
\newcommand\mainTitle{Vymyslet nadpis}
\newcommand\mainTitleEN{En Title}
\newcommand\authorName{Jakub Kacálek, Jan Volhejn}
\newcommand\consultant{}


\title {\mainTitle}
\author{\authorName}
\date{\publicationYear}

\usepackage[top=2.5cm, bottom=2.5cm, left=3.5cm, right=1.5cm]{geometry} %% okraje
\usepackage[czech]{babel}
\usepackage[utf8]{inputenc}
\usepackage[T1]{fontenc}
\usepackage{cmap}

\usepackage{graphicx}

\usepackage{subcaption}

\usepackage{hyperref}

\linespread{1.15}

\usepackage[pagestyles]{titlesec} %% balíček pro úpravu stylu kapitol a sekcí
\titleformat{\chapter}[block]{\scshape\bfseries\LARGE}{\thechapter}{10pt}{\vspace{0pt}}[\vspace{-22pt}]
\titleformat{\section}[block]{\scshape\bfseries\Large}{\thesection}{10pt}{\vspace{0pt}}
\titleformat{\subsection}[block]{\bfseries\large}{\thesubsection}{10pt}{\vspace{0pt}}

\setcounter{secnumdepth}{2}
\setcounter{tocdepth}{1}
\usepackage{fancyhdr}
\pagestyle{fancy}
\renewcommand{\headrulewidth}{1pt}

\usepackage{booktabs}

\usepackage{url}

%%%%%%%%%%%%%%%%%%%%%%%%%%%%%%%%%%%%%%

\usepackage{pdfpages}

\usepackage{upgreek}

\usepackage{amsmath}    %% Balíčky amsmath a amsfonts 
\usepackage{amsfonts}   %% pro sazbu matematických symbolů
\usepackage{esint}     %% pro sazbu různých integrálů (např \oiint)
\usepackage{mathrsfs}

%% makra pro sazbu matematiky
\newcommand{\dif}{\mathrm{d}} %% makro pro sazbu diferenciálu, místo toho
%% abych musel psát '\mathrm{d}' mi stačí napsat '\dif' což je mnohem 
%% kratší a mohu si tak usnadnit práci

\begin{document}

\pagestyle{empty}
\pagenumbering{Roman}

\begin{titlepage}
  \bfseries{
    \begin{center}
      \LARGE{STŘEDOŠKOLSKÁ ODBORNÁ ČINNOST}

      \vspace{14pt}
      \large{
        \specialization
      }

      \vspace{0.4 \textheight}

      \LARGE{
        \mainTitle
      }
      \vspace{0.4 \textheight}
    \end{center}

    \noindent\Large{\authorName}

    \noindent\Large{\district \hspace{\stretch{1}} \city, \publicationYear}
    }
\end{titlepage}

\cleardoublepage

{\bfseries
  \begin{center}
    \LARGE{STŘEDOŠKOLSKÁ ODBORNÁ ČINNOST}

    \vspace{14pt}
    {\large
      \specialization
    }

    \vspace{0.3 \textheight}

    \LARGE{
      \mainTitle
    }

    \LARGE{    
      \mainTitleEN
    }

    \vspace{0.24\textheight}
  \end{center}
}
{\Large
  \noindent\textbf{Autoři:} \authorName\\
  \textbf{Škola:} \school\\
  \textbf{Kraj:} \district\\
  \textbf{Konzultant: } \consultant\\
}

\noindent \city, \publicationYear

\cleardoublepage

\noindent{\Large{\bfseries{Prohlášení}}}                                                                              

\noindent Prohlašujeme, že jsme svou práci SOČ vypracovali samostatně a použili jsme pouze prameny a literaturu uvedené v seznamu bibliografických záznamů.

\noindent Prohlašujeme, že tištěná verze a elektronická verze soutěžní práce SOČ jsou shodné. 

\noindent Nemáme závažný důvod proti zpřístupňování této práce v souladu se zákonem č. 121/2000 Sb., o právu autorském, o právech souvisejících s právem autorským a o změně některých zákonů (autorský zákon) ve znění pozdějších předpisů. 

\vspace{24 pt}

\noindent V Pardubicích dne 9. září 3020 \dotfill{}\hspace{\stretch{0.5}} 

\hspace{5.75cm} \authorName

\cleardoublepage

\pagenumbering{arabic}
\pagestyle{fancy}
\setcounter{page}{1}

\end{document}